\documentclass[conference]{IEEEtran}
\IEEEoverridecommandlockouts
\usepackage{mystyle}
% The preceding line is only needed to identify funding in the first footnote. If that is unneeded, please comment it out.
%\usepackage{cite}
\usepackage{amsmath,amssymb,amsfonts}
\usepackage{algorithmic}
\usepackage{graphicx}
\usepackage{textcomp}
\def\BibTeX{{\rm B\kern-.05em{\sc i\kern-.025em b}\kern-.08em
    T\kern-.1667em\lower.7ex\hbox{E}\kern-.125emX}}
\begin{document}

\title{Automatic Parallelisation of Rust Programs at Compile Time}

\author{Michael Oultram}

\maketitle

\begin{abstract}
  There is a significant amount of research in automatic translation of sequential source code into a parallelized version. Most of this has been focused on FORTRAN and C which are both unsafe programming languages. This paper explores the literature and applies these ideas to the safe programming language Rust.
  \todo{Expand, and cleanup}
\end{abstract}

\section{Introduction}

\section{Literature Review}
Most research is for FORTRAN and the DO loops \parencite{Banerjee1993}.

\todo{Look at the different models, try to explain the differences}

Some people have converted C-to-CUDA \parencite{Baskaran2010, Verdoolaege2013}.

\section{Problem Details}

\printbibliography
\end{document}
