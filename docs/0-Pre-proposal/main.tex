\documentclass[11pt, a4paper]{article}
\usepackage{mystyle}

\title{\huge\textbf{Final Year Project}\vspace{-4ex}}
\date{}

\begin{document}
\maketitle

\section{Idea}
Combine the resources of two separate machines into one.
Each user-land process can access the resources of either machine.

\section{How it will be achieved}
I plan to use an existing open-source micro-kernel operating system (Redox).

I would change the process queue to be shared between the two kernels. When a 
kernel allows a process to run, it'll choose a process from this shared queue. 

Which kernel should run which process? If one kernel has a resource that the 
process wants, maybe that kernel should run the process and not the other one.

If a process is running that needs access to a resource on the

\section{Project Inspiration}
Moore's law is coming to an end (or so the internet says).

Consolidate all resources into one ``system".

Sounded cool

\section{Project Rough Plan}
\begin{itemize}
	\item Create a communication link between the two (or more) kernels
	
	\item Produce a distributed queue
	
	\item Change the process queue to this distributed queue instead. The 
	kernel will probably only take own processes at this point.
	
	\item Change resource mappings so there are no conflicts (e.g. kernel 
	A has eth0 and kernel B has eth1).
	
	\item Make it so if kernel A needs to access eth1, it uses the 
	communication link
	
	\item Each kernel can take any \textit{new} process from the queue. The 
	process should stay on the same kernel at this point(so that memory doesn't 
	need to be copied between machines)
	
	\item Allow an already started process to change kernel by copying the 
	processes memory.
	
	\item Deal with kernel drop-out somehow?
	
\end{itemize}

\end{document}
