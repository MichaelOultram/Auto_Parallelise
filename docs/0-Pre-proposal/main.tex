\documentclass[11pt, a4paper]{article}
\usepackage{mystyle}

\title{\huge\textbf{FYP Idea}\vspace{-4ex}}
\date{}

\begin{document}
\maketitle

Combine the resources of two separate machines into one. Each individual 
user-land process will be executed on one machine but it can access the 
resources of either machines. 

\textbf{Example}: Since this is a microkernel, the filesystem is in userspace. 
All the drives of all the machines will be visible to the filesystem. When the 
filesystem tries to access a drive, it sends the request to the kernel. If this 
drive is local the kernel can perform the action as normal. If the drive is 
remote, the kernel can forward the request to the other kernel.

The main ``in-depth" part of the project is designing the algorithm for 
choosing which process should run where. Accessing resources from a different 
kernel is likely to significantly slower than if the resource was local. So 
processes should be run on the machine that they use the most resources from. 
Problem: this isn't known before the program starts accessing resources. It may 
be possible to transfer execution of a process from one machine to another, but 
again this will take some time (have to copy memory).

\textbf{Project Rough Plan}
\begin{itemize}
	\item Download and compile Redox (an existing microkernel OS written in 
	Rust)
	
	\item Create a communication link between the two (or more) kernels
	
	\item Produce a distributed queue
	
	\item Change the process queue to this distributed queue instead. The 
	kernel will probably only take own processes at this point.
	
	\item Change resource mappings so there are no conflicts (e.g. kernel 
	A has eth0 and kernel B has eth1).
	
	\item Make it so if kernel A needs to access eth1, it uses the 
	communication link
	
	\item Each kernel can take any \textit{new} process from the queue. The 
	process should stay on the same kernel at this point(so that memory doesn't 
	need to be copied between machines)
	
	\item Allow an already started process to change kernel by copying the 
	processes memory.
	
\end{itemize}

\textbf{Extensions}: To simplify the project I will initially use two kernels 
which are connected together on a LAN. I'll assume that both kernels remain 
online, and the communication be relatively instantaneous. However, if the 
project needs extending, I could increase the number of running kernels and I 
could also look into how to resolve when a kernel disconnects unexpectedly.

\end{document}
