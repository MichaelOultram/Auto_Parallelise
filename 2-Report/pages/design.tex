\chapter{Design}
\label{chapter:Design}
This chapter covers the basic design concepts of my parallelising compiler without realising the full details of the compiler itself. Once I began developing the compiler, I realised some mistakes in the design which I did not fully think through and some adaptations that I had to do due to the structure of the rust compiler. These changes are described in \autoref{chapter:Implementation}. Also due to time constraints, not all the features described in this chapter were implemented.

Parallelising compilers described in the literature (\autoref{sec:related-work}) are split into several stages. Similarly, the design of my compiler is in two main stages, the analysis stage and the modification stage. Both these stages contain several steps to achieve their goal. The analysis stage looks at each function in the source code and tries to find parts that could be parallelised. The modification stage takes the parts that can be parallelised and changes the source code so that they run in parallel.

\section{Analysis Stage}
The rust abstract syntax tree consists of three main types: Blocks, Statements and Expressions. A block contains a list of statements and a statement is a combination of expressions. A block can be represented as an expression (either directly, or inside a loop/if/etc.) which allows for infinite depth in the tree. Variables are a type of expression. My parallelising compiler will focus on statement level parallelisation within a given block.

Each statement is analysed individually to provide a list of variables that the statement references. This list describes the variable dependencies that the statement has, but it does not describe which statements must be executed before the current one for the program to remain correct. To get this information, the algorithm looks at each statement in turn. For each variable that it requires, it looks at the statements before it in the block in reverse order for that variable to be its list of variable dependencies. The first statement found containing this variable as a dependency is added as a statement dependency. Any variables that were not found above the current statement must be defined outside the current block, and so are added as the current blocks dependency.

\section{Modification Stage}
