\chapter{Design}
\label{chapter:Design}
This chapter covers the original design concepts of my parallelising compiler without realising the full internal details of the compiler. Once I began developing the compiler, I realised some mistakes in the design which I did not fully think through and some adaptations that I had to do due to the structure of the rust compiler. These changes are described in \autoref{chapter:Implementation}. Also due to time constraints, not all the features described in this chapter were implemented.

The rust abstract syntax tree consists of three main types: Blocks, Statements and Expressions. A block contains a list of statements and a statement is a combination of expressions. A block can be represented as an expression (either directly, or inside a loop/if/etc.) which allows for infinite depth in the tree. Variables are a type of expression. My parallelising compiler will focus on statement level parallelisation within a given block.

Parallelising compilers described in the literature (\autoref{sec:related-work}) are split into several stages. Similarly, the design of my compiler is in two main stages, the analysis stage and the modification stage. Both these stages contain several steps to achieve their goal. The analysis stage looks at each function in the source code and tries to find parts that could be parallelised. The modification stage takes the parts that can be parallelised and changes the source code so that they run in parallel.

\section{Analysis Stage}
\begin{algorithm}[H]
\caption{Dependency Analysis Algorithm}
\label{alg:dependency-analysis}
\begin{algorithmic}[1]
    \Require $block$ as a list of Statements
    \State $envs = \{\}$
    \State $deps = \{\}$
    \State $block\_env = []$ \Comment Statement Environment of the Block when represented as a Statement
	\For{$stmt$ \textbf{in} $block$}
        \State $stmt\_env =$ Set of variables that $stmt$ uses
        \State $envs.push(stmt\_env)$
        \State $stmt\_deps = \{\}$
        \For{$i =$ length($stmt\_env$) $- 1$ \textbf{to} 0} \Comment Search backwards for $var$s in $stmt\_env$
            \State $dep\_env = envs[i]$
            \For{\textbf{all} $var$ \textbf{where} ($var$ \textbf{is in} $stmt\_env$) \textbf{and} ($var$ \textbf{is in} $dep\_env$)}
                \State $stmt\_deps.push(i)$
                \State $stmt\_env.remove(var)$
            \EndFor
        \EndFor
        \State $block\_env.push(stmt\_env)$ \Comment Unmatched variables are part of the block's environment
    \EndFor
\end{algorithmic}
\end{algorithm}

Each statement is analysed individually to provide a list of variables that the statement accesses. This list describes the variable dependencies that the statement has, but it does not describe which statements must be executed before the current one for the program to remain correct. To get this information, the algorithm looks at each statement in turn. For each variable that it requires, it looks at the statements before it in the block in reverse order for that variable to be its list of variable dependencies. The first statement found containing this variable as a dependency is added as a statement dependency. Any variables that were not found above the current statement must be defined outside the current block, and so are added as the current blocks dependency. This algorithm is shown in Algorithm \autoref{alg:dependency-analysis}.


\section{Modification Stage}
% Takes the DependencyTree of a function with all relative dependencies added.
% Converts the relative dependencies into StmtIDs
% Maximum Spanning Tree:
% 	Looks for any fully independent statements and starts a new ScheduleTree for each of them
% 	while remaining_nodes:
%		For all remaining nodes:
%			if the node has all it's dependencies on the tree already
%				find dependency node with maximum performance value (+ all dependent nodes on tree)
%				add node to tree here (with a preresquite of all the other dependencies)
%				add synclines onto the other dependency nodes so that all dependencies will be met
% Inside a block, a new Schedule is created

% Performance Metric:
% Currently just a fixed number 1
The dependency tree provided by the analysis stage shows what statements can be run in parallel. Some of these statements have multiple dependencies, all of which must be met before the statement is run. Each of these dependencies could be in a separate thread, and so some synchronisation technique is needed. The dependency tree is converted into a schedule tree so that we know which statements are run in which threads, and where/when synchronisation is required between threads.

\begin{algorithm}[H]
\caption{Scheduling Algorithm}
\label{alg:scheduling}
\begin{algorithmic}[1]
    \Require $block$ as a list of Statements
    \Require $envs$ as a list Statement Environments
    \Require $deps$ as a list Statement Dependencies
    \State $schedule\_trees = []$
    \For{$i=0$ \textbf{in} length($block$) - 1} \Comment Add all independent statements to separate trees
        \If{length($deps[i]$) == 0}
            \State $schedule\_trees.push((``Run", i, []))$
        \EndIf
    \EndFor
    \While{\textbf{not all} Statements \textbf{from} $block$ \textbf{are in} $schedule\_trees$}
        \For{$stmtid = 0$ \textbf{to} length($blocks$) - 1 \textbf{where} $block[stmtid]$ \textbf{is not in} $schedule\_trees$}
            \If{\textbf{all} $deps[stmtid]$ \textbf{are in} $schedule\_trees$}
                \State $dep\_trees =$ \textbf{find all} $(``Run", i, \_)$ \textbf{in} $schedule\_trees$ \textbf{for all} $i$ \textbf{in} $deps[stmtid]$
                \State \textbf{sort descending} $dep\_trees$ \textbf{and} $deps[stmtid]$ \textbf{by} depth \textbf{in} $schedule\_trees$
                \For{$i = 1$ \textbf{to} length($dep\_trees$)}
                    \State $(\_,\_,subtrees) = dep\_trees[i]$
                    \State $subtrees.push((``SyncTo", stmtid))$
                \EndFor
                \State $(\_,\_,subtrees) = deps[0]$
                \State $subtrees.push((``SyncFrom", deps[stmtid][1:], [(``Run", stmtid, [])])$
            \EndIf
        \EndFor
    \EndWhile
\end{algorithmic}
\end{algorithm}

The scheduling algorithm designed aims to run as much as possible in separate threads. It makes the naive assumption that threads have no overhead. To start of with, the algorithm looks for any statements with no dependencies. Each of these statements can be run in a separate thread. For all the remaining statements, the algorithm looks selects the set of statements that have all their dependencies in the schedule already. If the statement only has one dependency, then that statement is added directly after that dependency. If the statement has more than one, then the algorithm looks to see which dependency has the longest chain. The thought behind this is that this dependency should be the slowest, and so all the other dependencies should have been finished by this point. To make sure that there are no race conditions, a syncline is introduced from the other dependencies to just before the current statement. This algorithm is repeated until all the statements are in the schedule. Since there cannot be any cyclic dependencies due to the way that the dependency analysis algorithm was designed, this is guaranteed to terminate. This algorithm is shown in Algorithm \autoref{alg:scheduling}.
